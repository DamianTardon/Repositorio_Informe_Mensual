\documentclass[11pt]{report} % Clase de documento para un informe
% Documento y codificación general
\usepackage[utf8]{inputenc} % Tildes, "ñ" y símbolos especiales.
% Idioma y traducción
\usepackage[spanish,es-tabla]{babel} % Traduce el documento al español.
\usepackage[a4paper,left=3cm,right=2.5cm,top=2.5cm,bottom=2.5cm]{geometry} % Permite cambiar el tamaño de la página.
% Imágenes y figuras
\usepackage{graphicx} % Permite incluir imágenes.
\usepackage{subcaption} % Permite agregar subfiguras dentro de una figura principal, cada una con su leyenda.
\usepackage{float} % Permite fijar la posición de las figuras y tablas [H].
% Matemáticas y símbolos
\usepackage{amsmath} % Permite usar entornos matemáticos avanzados.
\usepackage{amssymb} % Permite usar símbolos matemáticos adicionales.
\usepackage{mathtools} % Permite usar herramientas matemáticas avanzadas.
% Hipervinculos y referencias
\usepackage[hidelinks]{hyperref} % Permite crear hipervínculos transparentes dentro del documento.
\usepackage[table,xcdraw]{xcolor} % Permite usar colores textos y tablas.
\usepackage{setspace} % Permite ajustar el interlineado.


\begin{document}
    \begin{spacing}{1.5}
        \begin{flushleft}
            
                \begin{center}
                    \Large
                    \textbf{FORMULARIO DE AVANCE MENSUAL DE\\
                    PROYECTO INTEGRADOR}
                \end{center}
                \vspace{2em}
                
                FECHA: 03/11/2025

                ALUMNO: Tardón, Damián David

                TEMA: Diseño, desarrollo e implementación de un software para la adquisición y análisis de ondas de impulso tipo rayo (1,2/50 \(\mu\)s).
                
                DIRECTOR: Blasco, Marcos

                CO-DIRECTOR: Serra, Gabriel Horacio.

                FECHA ESTIMADA DE FINALIZACIÓN: Marzo 2026.
                
                \hrulefill
                
                INFORME:
            
        \end{flushleft}
    \end{spacing}

    Se estudió las normas para el desarrollo del trabajo, se determinó la adquisición de las normas IEC 60060-1 e IEC 61083-2, y el LAT efectuó su compra. Luego, se instaló y probó el software Test Data Generator incluido con la norma IEC 61083-2, que provee herramientas para la generación de datos de prueba, que se utilizarán más adelante para validar la etapa de procesamiento de datos.

    Se realizó un análisis de los osciloscopios disponibles para este proyecto, con el fin de determinar cual resulta más adecuado considerando sus características técnicas: ancho de banda, tasa de muestreo y tensión máxima de entrada. Así como también la información complementaria suministrada por el fabricante, como los manuales de usuario, manuales de programación del instrumento, controladores USB y controladores para LabVIEW.

    Se determinó desarrollar el control del osciloscopio en el lenguaje de programación Python, utilizando la librería PyVISA, debido al elevado costo de la licencia de LabVIEW y considerando que, para los fines de este proyecto, las herramientas de Python permiten realizar las mismas tareas sin costo.

    Se desarrolló el módulo de control del osciloscopio. Actualmente, se dispone de un script que permite configurar el instrumento, adquirir la señal registrada, almacenarla y graficarla temporalmente en la computadora.

    Las actividades descriptas en este informe se corresponden con las tareas 1, 2 y 3 del diagrama de Gantt propuesto en la SAT.

\end{document}